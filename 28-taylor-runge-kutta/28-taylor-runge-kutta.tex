\documentclass[12pt]{article}
\usepackage[top=1in, bottom=1in, left=1in, right=1in]{geometry}

\usepackage{setspace}
\onehalfspacing

\usepackage{amssymb}
%% The amsthm package provides extended theorem environments
\usepackage{amsthm}
\usepackage{epsfig}
\usepackage{times}
\renewcommand{\ttdefault}{cmtt}
\usepackage{amsmath}
\usepackage{graphicx} % for graphics files

% Draw figures yourself
\usepackage{tikz} 

% The float package HAS to load before hyperref
\usepackage{float} % for psuedocode formatting
\usepackage{xspace}

% from Denovo Methods Manual
%\usepackage{mathrsfs}
%\usepackage[mathcal]{euscript}
%\usepackage{color}
%\usepackage{array}

\usepackage[pdftex]{hyperref}
\usepackage[parfill]{parskip}

% math syntax
\newcommand{\nth}{n\ensuremath{^{\text{th}}} }
\newcommand{\ve}[1]{\ensuremath{\mathbf{#1}}}
\newcommand{\Macro}{\ensuremath{\Sigma}}

%---------------------------------------------------------------------------
\title{NE 155, Class 28, S15 \\
Taylor Series Methods and Runge-Kutta}
\date{April 3, 2015}
\begin{document}
\author{guest lecturer: Kathryn Huff}
\maketitle

\hrulefill

\section{Introduction}

This lecture will cover:

\begin{itemize}
\item A reminder of the PRKE we learned yesterday
\item A derivation of Forward Euler for approximating solutions to initial value problems
\item A way to step through time with Runge-Kutta methods
\item And application of this to a PRKE simulation
\item An exploration of the importance of delayed neutrons for reactor control
\end{itemize}

\section{PRKE}
Point Reactor Kinetics Equations (PRKE).

\begin{align}
  \frac{dn(t)}{dt} &= \frac{\rho(t)-\beta}{\Lambda}n(t) + \sum_{i=1}^{6}\lambda_iC_i(t)\\
  \frac{dC_i(t)}{dt} &= \frac{\beta_i}{\Lambda}n(t) - \lambda_i C_i(t)
  \label{prke}
  \intertext{where}
  i&\in[1,6]\nonumber\\
  n &= \mbox{neutron population}\\
  \beta &= \mbox{fraction of neutrons that are delayed}\\
  \lambda_i &= \mbox{effective decay constant of the ith precursor}[\frac{1}{s}]\\
  C_i(t) &= \mbox{delayed neutron concentration due to the ith precursor}\\
  \Lambda &= \mbox{effective neutron lifetime}\\
  &\equiv (v\nu\Sigma_F)^{-1}\nonumber\\
  \rho(t) &= \mbox{reactivity}\\
  &\equiv\frac{k(t)-1}{k(t)}\nonumber\\
         &\equiv\frac{\nu\Sigma_F-\Sigma_a(1+L^2B_g^2)}{\nu\Sigma_F}\nonumber\\
  \intertext{and}
  k&\equiv \frac{\nu\Sigma_F/\Sigma_a}{1+L^2B_g^2}.\nonumber
\end{align}

The PRKEs allow a nuclear engineer to remove the spatial aspects of the
reactor from consideration to simplify the analysis of this initial value
problem. In particular, the assumptions that have gone into this set of
equations include:

\begin{itemize}
\item there is no external neutron source
\item One energy group ($\phi(\vec{r},E,t) = \phi(\vec{r},t)$)
\item Separation of variables ($\phi(\vec{r},t) = S(t)\psi(\vec{r})$)
\item $\beta_i$ and $\lambda_i$ are constant over the transient
\item There is no feedback (e.g. flux is low, feedbacks are slow/weak, $\rho(t)$ is known)
\end{itemize}

%BCs: $n(0) = n_0$ and $C_i(0) = C_{i,0}$ for $i=1,\dots,N$.

\paragraph{Exercise:}
\emph{What initial conditions need to be defined in order to solve this initial
value problem (eqn. \eqref{prke}?)}

\section{IVP}

Recall that the initial value problem takes the form:
\begin{align}
u'(t) &= f(u(t), t), \mbox{ for } t>t_0
\label{ivp}
\intertext{where}
u(t_0) &= u_0
\end{align}
In such a problem, we desire to compute $u(t_1)$, $u(t_2)$, $u(t_n)$ and so on. 

There are many methods for solving initial value problems numerically. 
This lesson will introduce a simple one, Forward Euler, which is derived from a 
Taylor series expansion. We will then use Forward Euler to introduce a two-stage, explicit 
Runge-Kutta method as well, for higher accuracy.  


%------------------------------------------------------------------------------
\section{Taylor Series Derivation of Forward Euler}

The simplest method for finding $u(t_n)$ is Forward Euler, which approximates 
$u(t_n)$. Let's call this approximation, $U^n$. In this notation, Forward Euler 
is based on replacing $u'(t_n)$ with $(U^{n+1} - U^n)/\Delta t$, where $\Delta t$ is the width of the timestep.
The Forward Euler method arises from a Taylor series expansion of $u(t_{n+1})$ 
about $u(t_n)$:

\begin{align}
u(t_{n+1}) = u(t_n) + \Delta tu'(t_n) + \frac{1}{2}\Delta t^2u''(t_n) + \cdots
\label{taylor}
\end{align}

With this, the $O(\Delta t^2)$ terms can be dropped to give:
\begin{align}
u(t_{n+1}) \approx u(t_n) + \Delta tu'(t_n) 
\end{align}

And, based on equation \eqref{ivp} we can replace $u'(t_n)$ with 
$f(u(t_n),t_n)$:

\begin{align}
\frac{1}{2}\Delta t^2u''(t_n) + \cdots\\
u(t_{n+1}) = u(t_n) + \Delta tf(u(t_n),t_n)
\end{align}

This expression gives a truncation error of order $O(\Delta t^2)$. More 
accurate schemes can be derived with a Taylor series expansion by retaining 
higher order terms in equation \eqref{taylor}. Since we are only given 
$u'(t_n) = f(u(t_n),t_n)$, however, the computation of such schemes requires 
repeated recursive differentiation of this function, and can get quite messy.


\section{Runge-Kutta Methods}

Runge-Kutta is a method used in practice to get a higher order approximation 
\emph{without} explicitly calculating higher order derivatives.

Runge-Kutta uses two stages. The first stage is an update using Euler's method, 
approximating $u(t_{n+1/2})$. 

\begin{align}
U^{n+1/2} = U^n + \frac{1}{2}\Delta tf(U^n)\\
\end{align}

The second stage evaluates the function, $f$, at the midpoint to estimate the 
slope.

\begin{align}
U^{n+1} = U^n + \Delta tf(U^{n+1/2})
\end{align}

These equations can be combined into a single expression:

\begin{align}
U^{n+1} = U^n + \Delta tf(U^n + \frac{1}{2}\Delta tf(U^n))
\end{align}

This approximation, because it uses two points, like a centered approximation, 
is order $O(\Delta t)$ accurate.

A generic r-stage Runge-Kutta method can be expressed as:
\begin{align}
Y_1 = U^n + \Delta t\sum_{j=1}^r a_{1j}f(Y_j, t_n + c_j\Delta t)\\
Y_2 = U^n + \Delta t\sum_{j=2}^r a_{2j}f(Y_j, t_n + c_j\Delta t)\\
\vdots\nonumber\\
Y_r = U^n + \Delta t\sum_{j=r}^r a_{rj}f(Y_j, t_n + c_j\Delta t)\\
U^{n+1} = U^n + \Delta t\sum_{j=r}^r a_{rj}f(Y_j, t_n + c_j\Delta t)
\end{align}



\section{Application to PRKE}

Each of these can be applied to the PRKE. In particular, let's consider the 
application of a Forward Euler.

To avoid confusion with the multiplication factor, the width of our timestep will be called $\Delta t$.

\begin{align}
n(t_{n+1}) = n(t) + \Delta t\left[\frac{\rho(t_n)-\beta}{\Lambda}n(t_n) + \displaystyle\sum^{j=J}_{j=1}\lambda_j\zeta_j\right]\\
\end{align}

\section{Delayed Neutrons and Reactor Control}
These delayed neutrons are critical to controlling the reactor.
To capture the reasons why, we will need the following definitions.

\begin{align}
\rho &= \mbox{reactivity}\\
&= \frac{k-1}{k}\\
k &= \mbox{multiplication factor}\\
&(k < 1) \rightarrow \mbox{negative reactivity}\\
&(k > 1) \rightarrow \mbox{positive reactivity}\\
&(k = 1) \rightarrow \mbox{critical}\\
\beta &= \mbox{delayed neutron fraction}\\
&(\rho < \beta) \mbox{ delayed supercriticality}\\
&(\rho > \beta) \mbox{ prompt supercriticality}\\
l &= \mbox{mean neutron lifetime}
\end{align}

\subsection{Units of Reactivity}
Note that the units of $\rho$ can be confusing.

    \begin{table}[h!]
    \centering
      \begin{tabular}{|l|c|c||}
        \hline
        Unit & Definition & Example\\
        \hline
        $\Delta k$ & actual PRKE units & 0.0005 \\
        $\%\Delta k$ & percent notation of $\Delta k$ & 0.05\% \\
        pcm & per cent mille & 50pcm \\
        \hline
        Dollars & $\frac{\Delta k}{\beta}$ & \$1\\
        Cents & 100 cents per dollar & 100 cents\\
        Milli-beta & 1000 milli-beta per dollar& 1000 milli-beta\\
        \hline
      \end{tabular}
      \caption{Common units of reactivity.}
      \label{tab:rho_units}
    \end{table}


\subsection{Thought Experiment}
If there were no delayed neutrons, then the time constant for power increase
would be approximately  $l_p$, the prompt neutron lifetime. That isn't the
case, but if it were, the reactor power would proceed thus:

\begin{align}
l &= \mbox{mean generation time}\\
n(t+l) &= n(t) + l\frac{dn}{dt} = k_\infty n(t)
\intertext{such that}
\frac{dn}{dt} &= \left(\frac{k-1}{l}\right)n(t)
\intertext{which gives}
n(t) &= n_0e^{\frac{(k-1)t}{l}}
\intertext{characterized by the time constant}
T &= \mbox{reactor period}\\
  &= \frac{l}{k-1}
\end{align}

In a universe without delayed neutrons, the mean neutron lifetime ($l$) would be
the prompt neutron lifetime ($l_p$).  Noting that the prompt neutron lifetime is about
$2\times10^{-5}s$, take a moment to think about the implications of this.

\paragraph{Exercise}
\emph{If a control rod were moved to introduce an excess reactivity of }$0.0005\Delta
k$\emph{, what would the power be one second later?}

\end{document}
